\documentclass[a4paper,oneside]{article}

\usepackage[utf8]{inputenc}
\usepackage[T2A]{fontenc}
\usepackage[english,russian]{babel}

\usepackage{amsmath}
\usepackage{mathtools}
\usepackage{amsfonts}
\usepackage{enumitem}
\usepackage{multirow}
\usepackage{amsthm}
\usepackage{minted}
\setminted{fontsize=\small, breaklines=true, style=emacs, linenos}
\usepackage[
  top=2cm, bottom=2cm,
  left=2cm, right=2cm,
]{geometry}
\usepackage{graphicx}
\graphicspath{ {./images/} }
\usepackage{float}

\newtheorem{theorem}{Теорема}[subsection]
\newtheorem*{theorem*}{Теорема}

% --- Определение --- %
\theoremstyle{definition}
\newtheorem{definition}{Определение}[subsection]
\newtheorem*{definition*}{Определение}
% ------------------- %

\title{{Теория кодирования и сжатия информации}\\{Лабораторная работа №9}}
\author{Гущин Андрей, 431 группа, 1 подгруппа}
\date{\the\year{} г.}

\begin{document}

\maketitle

\section{Сравнение коэффициента сжатия данных}


\begin{table}[H]
  \small
  \centering
  \begin{tabular}{|p{1.5cm}|p{1.5cm}|p{1.5cm}|p{1.5cm}|p{1.5cm}|p{1.5cm}|p{1.5cm}|p{1.5cm}|p{1.5cm}|}
    \hline
    \multirow{2}{1.5cm}{Исходный размер, байт} & \multicolumn{8}{c|}{Коэффициент сжатия данных} \\ \cline{2-9}
    & Код Хаффмана & Код Фано & Алг. Шеннона & Адп. код Хаффмана &
    Стопка книг & Алг. Гилберта-Мура & LZ77 & LZ78 \\ \hline \hline

    2      & 0.28571 & 0.28571 & 0.15385 & 0.5 & 0.4 & 0.15385 & 0.25 & 0.33333 \\ \hline
    33     & 0.3587 & 0.3587 & 0.17188 & 0.6 & 0.45205 & 0.16837 & 0.25 & 0.33333 \\ \hline
    2739   & 1.37087 & 1.34993 & 1.23267 & 1.43403 & 0.87873 & 1.06825 & 15.5625 & 2.23227 \\ \hline
    330    & 6.875 & 6.875 & 6.11111 & 5.07692 & 7.17391 & 3.47368 & 20.625 & 3.05556 \\ \hline
    59     & 0.35119 & 0.34911 & 0.17251 & 0.5619 & 0.4958 & 0.16905 & 0.25 & 0.33333 \\ \hline
    7958   & 1.01466 & 0.98102 & 1.25481 & 1.4075 & 1.30096 & 1.08464 & 1.07193 & 1.03579 \\ \hline
    138245 & 1.14738 & 1.12422 & 1.51266 & 1.61462 & 1.32241 & 1.27212 & 1.31687 & 1.31643 \\ \hline
    574426 & 1.11119 & 1.12263 & 1.57949 & 1.69144 & 1.41842 & 1.31906 & 1.38212 & 1.41673 \\ \hline
    2752   & 7.88539 & 7.88539 & 7.81818 & 7.95376 & 7.90805 & 7.81818 & 57.33333 & 12.3964 \\ \hline
    2814   & 7.64674 & 7.64674 & 6.76442 & 7.64674 & 7.62602 & 3.66406 & 50.25 & 8.01709 \\ \hline
  \end{tabular}
\end{table}

\section{Сравнение скорости сжатия данных}

\begin{table}[H]
  \small
  \centering
  \begin{tabular}{|p{1.5cm}|p{1.5cm}|p{1.5cm}|p{1.5cm}|p{1.5cm}|p{1.5cm}|p{1.5cm}|p{1.5cm}|p{1.5cm}|}
    \hline
    \multirow{2}{1.5cm}{Исходный размер, байт} & \multicolumn{8}{c|}{Скорость сжатия данных, Кб/Сек} \\ \cline{2-9}
    & Код Хаффмана & Код Фано & Алг. Шеннона & Адп. код Хаффмана &
    Стопка книг & Алг. Гилберта-Мура & LZ77 & LZ78 \\ \hline \hline

    2      & 0.603018 & 0.715146 & 0.839344 & 0.061427 & 0.360548 & 0.431158 & 0.402022 & 0.381485 \\ \hline
    33     & 12.11943 & 14.134477 & 14.08 & 0.95642 & 13.245272 & 12.395048 & 11.545913 & 13.466972 \\ \hline
    2739   & 1115.8687 & 960.11501 & 982.9969 & 14.834143 & 932.34804 & 1053.4219 & 852.1148 & 987.4092 \\ \hline
    330    & 129.533301 & 139.46347 & 137.89838 & 9.618171 & 120.24553 & 140.288531 & 138.8474 & 136.0386 \\ \hline
    59     & 21.931573 & 21.791163 & 24.00079 & 1.522494 & 23.908192 & 23.849206 & 23.8374 & 23.762439 \\ \hline
    7958   & 2439.2702 & 2572.4858 & 2591.5064 & 8.141355 & 2292.5846 & 2635.5084 & 356.2595 & 629.98334 \\ \hline
    138245 & 16610.0 & 16047.029 & 14122.042 & 6.929059 & 7049.944 & 13377.705 & 177.1079 & 162.4453 \\ \hline
    574426 & 20524.878 & 17966.3016 & 18472.2615 & 9.0 & 9697.55 & 17911.865 & 170.0799 & 154.0666 \\ \hline
    2752   & 985.848522 & 1037.5729 & 911.10507 & 68.337438 & 1006.53558 & 821.1095 & 738.9178 & 789.0928 \\ \hline
    2814   & 1164.8452 & 1175.8971 & 1145.8538 & 63.616386 & 1141.20237 & 1044.0347 & 306.5708 & 1086.552 \\ \hline
  \end{tabular}
\end{table}

\end{document}
