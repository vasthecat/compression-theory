\documentclass[a4paper,oneside]{article}

\usepackage[utf8]{inputenc}
\usepackage[T2A]{fontenc}
\usepackage[english,russian]{babel}

\usepackage{amsmath}
\usepackage{mathtools}
\usepackage{amsfonts}
\usepackage{enumitem}
\usepackage{multirow}
\usepackage{amsthm}
\usepackage{minted}
\setminted{fontsize=\small, breaklines=true, style=emacs, linenos}
\usepackage[
  top=2cm, bottom=2cm,
  left=2cm, right=2cm,
]{geometry}
\usepackage{graphicx}
\graphicspath{ {./images/} }
\usepackage{float}

\newtheorem{theorem}{Теорема}[subsection]
\newtheorem*{theorem*}{Теорема}

% --- Определение --- %
\theoremstyle{definition}
\newtheorem{definition}{Определение}[subsection]
\newtheorem*{definition*}{Определение}
% ------------------- %

\title{{Теория кодирования и сжатия информации}\\{Лабораторная работа №13}}
\author{Гущин Андрей, 431 группа, 1 подгруппа}
\date{\the\year{} г.}

\begin{document}

\maketitle

\section{Сравнение коэффициента сжатия данных}

\begin{table}[H]
  \small
  \centering
  \begin{tabular}{|p{1.5cm}|p{2cm}|p{2cm}|p{2cm}|}
    \hline
    \multirow{2}{1.5cm}{Исходный размер, байт} & \multicolumn{3}{c|}{Коэффициент сжатия данных} \\ \cline{2-4}
                                               & RLE & LZW & Метод фрактального сжатия \\ \hline \hline
    196748  & 0.94165  & 0.80621 & \\ \hline
    196748  & 0.9534   & 0.82246 & \\ \hline
    196748  & 0.9729   & 0.81185 & \\ \hline
    196748  & 1.01265  & 0.75546 & \\ \hline
    786572  & 0.96528  & 0.60234 & \\ \hline
    786572  & 0.97416  & 0.67487 & \\ \hline
    786572  & 0.94578  & 0.75671 & \\ \hline
    786572  & 0.98144  & 0.80226 & \\ \hline
    65670   & 0.96401  & 0.81079 & \\ \hline
    65670   & 0.9502   & 0.51473 & \\ \hline
    65670   & 4.87311  & 4.73161 & \\ \hline
    65670   & 0.96234  & 0.79388 & \\ \hline
    262278  & 0.97824  & 0.95731 & \\ \hline
    1048710 & 0.96013  & 0.77848 & \\ \hline
    1048710 & 0.96401  & 0.8432  & \\ \hline
    262278  & 0.96022  & 0.71513 & \\ \hline
    262278  & 42.59834 & 20.2359 & \\ \hline
    786572  & 0.97923  & 0.78098 & \\ \hline
    262278  & 4.62629  & 19.5394 & \\ \hline
  \end{tabular}
\end{table}


\section{Сравнение скорости сжатия данных}

\begin{table}[H]
  \small
  \centering
  \begin{tabular}{|p{1.5cm}|p{2cm}|p{2cm}|p{2cm}|}
    \hline
    \multirow{2}{1.5cm}{Исходный размер, байт} & \multicolumn{3}{c|}{Скорость сжатия данных, Кб/Сек} \\ \cline{2-4}
                                               & RLE & LZW & Метод фрактального сжатия \\ \hline \hline
    196748  & 30077.998283  & 248.287166  & \\ \hline
    196748  & 41768.415466  & 247.55961   & \\ \hline
    196748  & 41948.873458  & 243.073281  & \\ \hline
    196748  & 36899.258608  & 360.422808  & \\ \hline
    786572  & 75005.79485   & 227.360301  & \\ \hline
    786572  & 76122.268973  & 177.373586  & \\ \hline
    786572  & 79913.654926  & 233.287219  & \\ \hline
    786572  & 80382.198847  & 229.016439  & \\ \hline
    65670   & 16260.689155  & 198.873762  & \\ \hline
    65670   & 19121.654937  & 302.15081   & \\ \hline
    65670   & 19552.541979  & 528.927159  & \\ \hline
    65670   & 18992.044058  & 219.517214  & \\ \hline
    262278  & 50870.853679  & 196.204039  & \\ \hline
    1048710 & 100810.048345 & 226.04576   & \\ \hline
    1048710 & 104181.711819 & 197.723091  & \\ \hline
    262278  & 47361.049597  & 211.420316  & \\ \hline
    262278  & 72066.189575  & 3543.689348 & \\ \hline
    786572  & 76326.050366  & 231.739497  & \\ \hline
    262278  & 59872.411971  & 623.32429   & \\ \hline
  \end{tabular}
\end{table}

\section{Сравнение качества сжатия данных}

Два реализованных алгоритма являются алгоритмами сжатия без потерь, поэтому
выполнить сравнение не представляется возможным.

\section{Выводы}

Оба алгоритма увеличивают размер файла при сжатии за счёт метаданных, если сам
файл достаточно малого размера.

Можно заметить, что алгоритм RLE практически всегда немного увеличивает файл
при сжатии за счёт метаданных, если в файле нет частых повторений каких-то
цветовых компонент. При этом данная реализация алгоритма LZW также не
сжимает большинство файлов из-за способа хранения метаданных.

При этом алгоритм на несколько порядков быстрее алгоритма LZW как при создании
архивов, так и при их разжатии.

\end{document}
