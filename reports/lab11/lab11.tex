\documentclass[a4paper,oneside]{article}

\usepackage[utf8]{inputenc}
\usepackage[T2A]{fontenc}
\usepackage[english,russian]{babel}

\usepackage{amsmath}
\usepackage{mathtools}
\usepackage{amsfonts}
\usepackage{enumitem}
\usepackage{amsthm}
\usepackage{minted}
\setminted{fontsize=\small, breaklines=true, style=emacs, linenos}
\usepackage{graphicx}
\graphicspath{ {./images/} }
\usepackage{float}

\newtheorem{theorem}{Теорема}[subsection]
\newtheorem*{theorem*}{Теорема}

% --- Определение --- %
\theoremstyle{definition}
\newtheorem{definition}{Определение}[subsection]
\newtheorem*{definition*}{Определение}
% ------------------- %

\title{{Теория кодирования и сжатия информации}\\{Лабораторная работа №11}}
\author{Гущин Андрей, 431 группа, 1 подгруппа}
\date{\the\year{} г.}

\begin{document}

\maketitle

\section{Задача}

Разработать программу осуществляющую архивацию и разархивацию цифрового
изображения используя алгоритм Лемпеля --- Зива --- Велча (LZW). Программа
архивации и разархивации должны быть представлены отдельно и работать независимо
друг от друга. Определить для данного шифра характеристику 1, 2 и 3. К работе
необходимо прикрепить отчет и программный проект.

\section{Алгоритм}

Алгоритм LZW заключается в создании нового кода для всех встречающихся
последовательностей символов. Алгоритм использует динамический словарь.

Алгоритм состоит из следующих шагов:
\begin{enumerate}
  \item
    Прочитать первый символ. Так как он ни разу не встречался, добавить его
    в словарь.
  \item
    Далее необходимо найти наибольшую подстроку ps, уже находящуюся в словаре.
  \item
    Если строка не найдена, то просто добавить отдельный символ аналогично
    самому первому. Иначе записать найденную подстроку и добавить в словарь
    конкатенацию найденной записи и следующего символа.
\end{enumerate}


\section{Тестирование}

Для проверки программы были использованы тестовые изображения 4.1.04.tiff
(рис. \ref{fig:test_1}, \ref{fig:test_1_img}), 4.2.01.tiff (рис.
\ref{fig:test_2}, \ref{fig:test_2_img}) и ruler.512.tiff (рис. \ref{fig:test_3},
\ref{fig:test_3_img}). Можно заметить, что после распаковки архива полученный
файл совпадает с исходным.

\begin{figure}[H]
  \centering
  \includegraphics[width=0.9\textwidth]{test1.jpg}
  \caption{Сжатие файла 4.1.04.tiff}
  \label{fig:test_1}
\end{figure}

\begin{figure}[H]
  \centering
  \includegraphics[width=0.9\textwidth]{test1_img.jpg}
  \caption{Сравнение разжатого изображения с 4.1.04.tiff}
  \label{fig:test_1_img}
\end{figure}


\begin{figure}[H]
  \centering
  \includegraphics[width=0.9\textwidth]{test2.jpg}
  \caption{Сжатие файла 4.2.01.tiff}
  \label{fig:test_2}
\end{figure}

\begin{figure}[H]
  \centering
  \includegraphics[width=0.9\textwidth]{test2_img.jpg}
  \caption{Сравнение разжатого изображения с 4.2.01.tiff}
  \label{fig:test_2_img}
\end{figure}


\begin{figure}[H]
  \centering
  \includegraphics[width=0.9\textwidth]{test3.jpg}
  \caption{Сжатие файла ruler.512.tiff}
  \label{fig:test_3}
\end{figure}

\begin{figure}[H]
  \centering
  \includegraphics[width=0.9\textwidth]{test3_img.jpg}
  \caption{Сравнение разжатого изображения с ruler.512.tiff}
  \label{fig:test_3_img}
\end{figure}

\section{Вычисленные характеристики}

\subsection{Характеристика 1 (Коэффициент сжатия)}

Результаты применения программы к каждому из тестовых графических файлов занесены
в таблицу \ref{tbl:results}.

\begin{table}[H]
  \small
  \centering
  \begin{tabular}{|c|c|c|c|}
    \hline
    Название     & Исходный размер, байт & Сжатый размер, байт & Коэффициент \\ \hline \hline
    4.1.04.tiff     &  196748 &  244041 & 0.80621 \\ \hline
    4.1.05.tiff     &  196748 &  239220 & 0.82246 \\ \hline
    4.1.06.tiff     &  196748 &  242346 & 0.81185 \\ \hline
    4.1.08.tiff     &  196748 &  260436 & 0.75546 \\ \hline
    4.2.01.tiff     &  786572 & 1305870 & 0.60234 \\ \hline
    4.2.03.tiff     &  786572 & 1165509 & 0.67487 \\ \hline
    4.2.05.tiff     &  786572 & 1039467 & 0.75671 \\ \hline
    4.2.07.tiff     &  786572 &  980445 & 0.80226 \\ \hline
    5.1.09.tiff     &   65670 &   80995 & 0.81079 \\ \hline
    5.1.11.tiff     &   65670 &  127582 & 0.51473 \\ \hline
    5.1.13.tiff     &   65670 &   13879 & 4.73161 \\ \hline
    5.1.14.tiff     &   65670 &   82720 & 0.79388 \\ \hline
    5.2.10.tiff     &  262278 &  273973 & 0.95731 \\ \hline
    5.3.01.tiff     & 1048710 & 1347127 & 0.77848 \\ \hline
    5.3.02.tiff     & 1048710 & 1243720 & 0.8432  \\ \hline
    boat.512.tiff   &  262278 &  366754 & 0.71513 \\ \hline
    gray21.512.tiff &  262278 &   12961 & 20.2359 \\ \hline
    house.tiff      &  786572 & 1007154 & 0.78098 \\ \hline
    ruler.512.tiff  &  262278 &   13423 & 19.5394 \\ \hline
  \end{tabular}
  \caption{результаты тестирования}
  \label{tbl:results}
\end{table}

\subsection{Характеристика 2 (Скорость сжатия)}

Для тестирования скорости сжатия использовался произвольный графический
файл размера 4808956 байт ($\approx$4.6 мегабайта). В результате пяти
последовательных запусков, среднее время запаковки файла составило 13.25
секунды, среднее время распаковки составило 0.04 секунд.

Таким образом, средняя скорость сжатия составила 354.43367 Кбайт в секунду, а
средняя скорость разжатия составила 117406.15234 Кбайт в секунду (114.65445
Мбайт в секунду).

\subsection{Характеристика 3 (Качество сжатия)}

Качество изображения не изменилось после сжатия, так как этот алгоритм является
алгоритмом сжатия без потерь.

\section{Реализация}

Программа реализована на языке программирования Rust с использованием библиотеки
clap для чтения параметров командной строки, а также библиотеки tiff для
чтения и записи tiff файлов. Сборка производится с помощью программы cargo,
поставляющейся вместе с языком.

\subsection{Содержимое файла lzw.rs}
\inputminted{rust}{../../lab11/src/lzw.rs}

\subsection{Содержимое файла main.rs}
\inputminted{rust}{../../lab11/src/main.rs}

\subsection{Содержимое файла Cargo.toml}
\inputminted{toml}{../../lab11/Cargo.toml}

\end{document}
